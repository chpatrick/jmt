%
% This document is free; you can redistribute it and/or modify
% it under the terms of the GNU General Public License as published by
% the Free Software Foundation; either version 2 of the License, or
% (at your option) any later version.
%
% This document is distributed in the hope that it will be useful, but
% WITHOUT ANY WARRANTY; without even the implied warranty of
% MERCHANTABILITY or FITNESS FOR A PARTICULAR PURPOSE.  See the GNU
% General Public License for more details.
%
% You should have received a copy of the GNU General Public License
% along with this document; if not, write to the Free Software
% Foundation, Inc., 51 Franklin Street, Fifth Floor, Boston, MA
% 02110-1301, USA.
%
% Author: Bertoli Marco
%
\chapter{Basic Definitions}
\label{cha:glossary}
\begin{description}
\item[Customer :] or \emph{job} or \emph{transaction} is the element that will require
service from our network. For example, it can be an \emph{http
request}, a \emph{database query}, or an \emph{ftp download
command}.
\item[Class :] a group of elements that are statistically equal.
The customers of a class have the same workload intensity and the
same average service demands.
\item[Station :] or service center, it represents a system resource.
Customers arrive at the station and then, if necessary, wait in
queue, receive service from server then depart.
\item[Workload Intensity ($\lambda$ or $N$) :] rate at which customers of a
given class arrive at the system.  Each class has its own workload
intensity. If a class is closed the constant \emph{number of
customers} $N$ that are present in the system must be provided, if a
class is open the \emph{arrival rate} $\lambda$ of customers to the
system is requested.
\item[Population Mix ($\beta_i$) :] the ratio between \emph{number of
customers} of closed class $i$ and the total number of customers in
the system: $\beta_i = N_i / \sum_k N_k$
\item[Service Time ($S_k$):] average time of service required at
each visit to resource $k$; it is computed as the ratio of the busy
time to the number of system completions. It is an alternatively way
to describe the service requirement.
\item[Visit (number of -) ($V_k$) :] average number of visits that
a customer makes at station $k$ during a complete execution; it is
computed as the ratio of the number of completions at resource k to
the number of system completions. If resource $k$ is a delay center
representing a client station, it is a convention assign a unitary
value to number of visits to this station.
\item[Service Demand ($D_k$) :] average service requirement of
a customer, that is the total amount of service required  by a
complete execution at resource $k$. In the model it is necessary to
provide separate service demand for each pair service center-class.
It is given by $D_k = V_k * S_k$.
\item[Throughput ($X_k$) :] rate at which customers are executed
by station $k$, that is the number of completions in a time unit.
\item[Queue length ($Q_k$) :] average number of customers at station
$k$, either waiting in queue and receiving service.
\item[Response Time ($R_k$):] average time interval between the instant
a customer arrives at station $k$ and the instants it terminate its
servicing.
\item[Residence Time ($W_k$) :] average
time that a customer spent at station $k$ during a complete
interaction with the system. It includes time spent queueing and
time spent receiving service. It does not correspond to
\emph{Response Time} $R_k$ of a station since $W_k = R_k * V_k$.
\item[Utilization ($U_k$) :] proportion of time in which the station $k$ is busy or, in the case
of a delay center, is the average  number of  customers in the
station (see Little Law \cite{Little}).
\item[System Throughput ($X$) :] rate at which customers perform
an entire interaction with the system. This is the aggregate measure
of \emph{Throughput}: $X = X_k / V_k$.
\item[System Response Time ($R$):] correspond to the intuitive
notion of response time perceived by users, that is, the time
interval between the instant of the submission of a request to a
system and the instant the corresponding reply arrives completely at
the user. It is the aggregate measure of \emph{Residence Times}: $R
= \sum_k W_k$.
\item[Average number of customers in the system ($N$):] the average
number of customer in the system, either waiting in queue or
receiving service. It is the aggregate measure of \emph{Queue
Length}: $N = \sum_k Q_k$.
\end{description}
